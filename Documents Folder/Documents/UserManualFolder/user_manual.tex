\documentclass[12pt,a4paper]{article}
\usepackage{graphicx}
\begin{document}
	\begin{titlepage}
		\centering
		\vspace*{\fill}
		
		\vspace*{0.5cm}
		
		\huge\bfseries
		\rule{\textwidth}{1.6pt}\\[\baselineskip]
		Thutong Site Learning Center User Manual
		
		\vspace*{0.5cm}
		
		\large Contributors: \\[\baselineskip]
		
			{Fiwa Lekhulani\\Daniel Rocha\\lebogang Ntatleng\\Lesego Mabe\\Tlou Lebelo\\Oluwatosin Botti}
		
		\rule{\textwidth}{1.6pt}\\[\baselineskip]
		
		
		\vspace*{\fill}
	\end{titlepage}


	\date{\textbf{\today}}
	\pagenumbering{roman}
	%\noindent\rule{\textwidth}{1pt}
	\pagebreak
	\tableofcontents
	\newpage
	\pagenumbering{arabic}

s

	\section{Product Overview}
		The Thutong Site Learning Centre is an online learning system intended for high school students. It is aimed at providing students with online material to learn or catch up on any academic content that they might have missed in class or did not comprehend during class. it is also aimed at providing teachers a way of uploading academic content to the system for students to learn from and provide quizzes for the students to test their knowledge after the completion of a topic. 
	
	\section{System Configuration}
		The Thutong Site Learning Centre does need not be installed on any any digital device, as it is available online. The system can be accessed through desktop computers and mobile devices and users will need a network connection and mobile data in order to huse the learning centre 
		
		%NOTE:please load an image here!!!
		
	\section{Getting Started}
		The Thutong Learning Centre is aimed at improving South Africa's pass rate; it intendeds to be used by every student in South Africa to improve their understand and envitably pass. It is free and no license is needed to use it to enable more students to have access to it.\\
		There Thutong Learning Centre encompasses four types of users:
		\begin{itemize}
			\item Students
			\item Expert Consultants(Teachers)
			\item Marketing Consultants
			\item Administrator
		\end{itemize} 
		  
		\subsection{The student}
		 This student is able to login with either their google, facebook or email account in order to have personalized access when using the learning centre. The student is also able to search for a specific subject and be able to view content within that subject (regardless of their login status). Figure 1 shows the login page where the user will fill in their created account details to access the system. Figure 2 shows the registration form required to filled by new users\\
		 
		 %imagee to be loaded here(screenshot)
		 
		 \begin{figure}
		 	\includegraphics[width=\linewidth]{login.JPG}
		 	\caption{User Login.}
		 	\label{fig:user login}
		 \end{figure}
		 
		 \begin{figure}
		 	\includegraphics[width=\linewidth]{Register.JPG}
		 	\caption{User Registration.}
		 	\label{fig:user registration}
		 \end{figure}
		 
		 \subsection{The Expert Consultant}
		 The Expert consultant needs to be able to, other than logging in, add academic content intended for students. They should also be able to create quizzes for students and be able to remove all of the above-mentioned entities.
		
		\subsection{The Marketing Consultant}
		 This user needs to be able to add, remove and update marketing content within the system such as vacancies, advertisements and donation request.
		 
		 
		 \subsection{The Administrator}
		 Also known as the superuser, this user will have all the access to all users profilesbeing able to removed those deemed to use the learing cenre inappropriately. Admin will also be responsible for the aapproval, addition, validation and removal of Expert and Marketing consultant accounts.
		 
	\section{Using the System}
	 This section discusses the use cases of the system in detail, please refer to the subsections below for each use case by the various users of the system.
	 	\subsection{Login}
	 	 From 4.1, Figure 1 above, the user enters their credentials and  is welcomed by the home page which is Figure 2. This a a generic use case for all users and all users are welcomed by this page.
	 	 
	 	 \subsection{Search}
	 	 To search for a specific course or subject admin related content, the user clicks on the button on right written "search" in bold red text, they will then be taken to Figure 3 where they enter whatever topic they which to search for and select the subject are, and the result will be displayed right opposite the search bar as seen on Figure 4.
	 	 
	 	 
	 	 
	 	 \subsection{Viewing course content}
		Viewing a document, course notes or videos for all users is similar. Figure 5 displays the Biology course notes and a video that may be played. The video can simply be viewed by clicking on the video.
		
		\subsection{Taking a quiz}
		Taking a quiz after reading up on a particular subject can be achieved by clicking on the particular quiz they would like to take, for instance clicking on "Quiz1 Types of Reactions" link at the bottom of Figure 5 and this will take you to the page depicted by Figure 6. The user will then take a quiz by selecting the answers they deem correct and thereafter click submit answers button at the end of the quiz. Their result will then be displayed right below the previously mentioned button in red.  
		 
		\pagebreak
		
		
		
		 \begin{figure}
		 	\includegraphics[width=\linewidth]{Home.jpg}
		 	\caption{home page}
		 	\label{fig:home page}
		 \end{figure}
		 
		 \begin{figure}
		 	\includegraphics[width=\linewidth]{Search.jpg}
		 	\caption{Search}
		 	\label{fig:Searching for subject}
		 \end{figure}
	 
	 	\begin{figure}
	 		\includegraphics[width=\linewidth]{searchAndView.jpg}
	 		\caption{Search results}
	 		\label{fig:Searching for subject continued..}
	 	\end{figure}
 	
 		 \begin{figure}
 			\includegraphics[width=\linewidth]{viewLesson.jpg}
 			\caption{Take a lesson}
 			\label{fig:Taking a lesson}
 		\end{figure}
 	
 		 \begin{figure}
 			\includegraphics[width=\linewidth]{biology.jpg}
 			\caption{Taking a Quiz}
 			\label{fig:Taking a Quiz}
 		\end{figure}
 	
 	\section{Troubleshooting}
 	 This section will be implemented at a later stage of the project.
 	 
 	 
 	 	\section{Additional Information and Figures}
	
\end{document}
